\chapter*{Foreword}

The Standard Model has been remarkably successful in predicting the interactions 
of fundamental particles for over 70 years.
Despite this success, the Standard Model leaves several
important questions unanswered.
In particular, the Standard Model does not 
explain the presence of three generations of both leptons and quarks, nor
does it explain the similarity of the arrangements of leptons and quarks
under the electroweak interaction.

One possible motivation for these features stems from the
idea that the Standard Model is a low energy remnant of a larger,
more elegant theory, which includes leptoquarks: heavy bosons
that carry both lepton and baryon number and mediate a fundamental
interaction between leptons and quarks.  Leptoquarks have been
a topic of interest in the particle physics community for some time. 
This interest was perhaps never more intense than in 
1997, when the ZEUS and H1 collaborations at the HERA electron-proton ($ep$) collider
\nomenclature{$ep$}{electron-proton}
observed an excess of $e + p \rightarrow e + p + X$ events corresponding
to an electron-quark invariant mass of about 200 GeV
\cite{hera-excess-h1,hera-excess-zeus}.
This excess sparked a flurry of both theoretical and experimental investigation into leptoquarks.
While leptoquarks of this description would have been produced in large
numbers at the Tevatron proton-antiproton ($p\overline{p}$)
\nomenclature{$p\overline{p}$}{proton-antiproton}
collider, results from the 
CDF and D0 experiments definitively ruled out any possibility for leptoquarks
at a mass scale of 200 GeV in 1998 \cite{tevatron-refutation}.
Even so, leptoquarks remain an active area of theoretical and experimental research
and are an important benchmark in the search for new physics beyond the Standard Model.

This thesis describes a search for the pair production of scalar leptoquarks decaying to fermions of the first generation.
This search is performed using $\sqrt{s} = 7$ TeV proton-proton ($pp$) 
\nomenclature{$pp$}{proton-proton}
collisions at the Large Hadron Collider as measured 
by the Compact Muon Solenoid detector.
The data used were collected in 2011 and correspond to an integrated luminosity of \lumi.
Chapter \ref{ch:introduction} gives a brief overview of the Standard Model and 
an effective model of leptoquarks.
Chapters \ref{ch:experiment} and \ref{ch:reco} describe the experimental
apparatus used to conduct the search: the Large Hadron Collider and the 
Compact Muon Solenoid detector.
Chapters \ref{ch:analysis-strategy}-\ref{ch:limit} describe the search itself.
Chapter \ref{ch:conclusion} presents a conclusion and lists future prospects for the search.
