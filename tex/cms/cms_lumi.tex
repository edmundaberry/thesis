\subsection{Luminosity measurement}
\label{sec:lumi}

As discussed in Section \ref{sec:lhc-parameters}, instantaneous luminosity, $\mathcal{L}$, is
the ratio between the production rate in Hz for given physics process in a collider and the 
production cross section for that process.  Integrated luminosity, \Lint, 
refers to an integral of instantanenous luminosity over a period of time.
Instantaneous luminosity depends only on beam parameters,
and it is expressed in units of $\text{cm}^{-2}\text{s}^{-1}$.
In CMS, instantaneous luminosity is measured with the pixel tracker (see Section \ref{sec:tracker}),
using a method based on pixel cluster counting (PCC).  
\nomenclature{PCC}{Pixel cluster counting}
The following description of the PCC method
is taken from Reference \cite{cms-lumi-uncertainty}.

In the PCC method, an effective cross section for the formation
of pixel clusters is determined using results from a Van der Meer
scan.  That cross section, $\sigma_{\text{pixel}}$, is used
to determine the instantaneous luminosity for each 23.3 second
luminosity section of the 2011 physics sample.  The high voltage 
of the pixel tracker is only turned on during stable collisions,
and pixel data is not available when the CMS data acquisition framework is not running.
For these reasons, the PCC method is not useful for online luminosity monitoring.
The PCC method is, however, used for all integrated luminosity measurements
mentioned in this thesis.

There are roughly 66 million pixels in the pixel tracker.
About 200 pixel clusters are formed in each minimum bias interaction,
and there are about 5 pixels per pixel cluster.  This means that
even at high luminosities, the occupation of the pixel detector
is on the order of $1/1000$, and the probability of a given pixel
being hit by two different tracks is very low.
As a result, the number of pixel clusters per 
bunch crossing is expected to be very linear with the number
of interactions per bunch cross and therefore a good measure 
of instantaneous luminosity.

Within the LHC, each bunch crossing produces some number of $pp$ 
interactions, which in turn produce some number of pixel clusters.
If these interactions are recorded using zero-bias triggers, then
the mean number of pixel clusters per trigger is expressed by 
Equation \ref{eqn:lumi1}:
\begin{equation}
  \left<N_{\text{cluster}}\right> = \left<N_{\text{cluster/interaction}}\right> \left<N_{\text{interactions}}\right> \equiv \left<N_{\text{cluster/interaction}}\right> \mu
  \label{eqn:lumi1}
\end{equation}
where $\left<N_{\text{cluster}}\right>$ is the average number of pixel clusters formed per bunch crossing,
$\mu = \left<N_{\text{interactions}}\right>$ is the average number of $pp$ interactions in a bunch crossing, and
$\left<N_{\text{cluster/interaction}}\right>$ is the average number pixel clusters formed per $pp$ interaction.

$\mu$, the average number of $pp$ interactions in a bunch crossing, may also be expressed
as a function of the $pp$ interaction cross section, $\sigma_{\text{interaction}}$, the per-bunch
instantaneous luminosity, $\mathcal{L}$, and the LHC orbital frequency, $f$, using Equation \ref{eqn:lumi2}:
\begin{equation}
  \mu = \frac{\sigma_{\text{interaction}}}{f}\mathcal{L}
  \label{eqn:lumi2}
\end{equation}
The LHC orbital frequency, $f$, is shown in Section \ref{sec:lhc-parameters} to be 11.246 kHz.

Equations \ref{eqn:lumi1} and \ref{eqn:lumi2} may be combined as follows in Equation \ref{eqn:lumi3}:
\begin{equation}
  \left<N_{\text{cluster}}\right> = \left<N_{\text{cluster/interaction}}\right> \frac{\sigma_{\text{interaction}}}{f}\mathcal{L}
  \label{eqn:lumi3}
\end{equation}
If $\sigma_{\text{pixel}}$ is defined to be equal to the product of 
$\sigma_{\text{interaction}}$ and $\left<N_{\text{cluster/interaction}}\right>$,
then Equation \ref{eqn:lumi3} may be rewritten as:
\begin{equation}
  \left<N_{\text{cluster}}\right> = \frac{\sigma_{\text{pixel}}}{f}\mathcal{L}
  \label{eqn:lumi4}
\end{equation}
This allows $\sigma_{\text{pixel}}$ to be expressed in terms of 
$\left<N_{\text{cluster}}\right>$, $f$, and $\mathcal{L}$: all of which
are measureable during a Van der Meer scan, as was documented in Reference \cite{cms-lumi-uncertainty}.
The relevant relationship is given by Equation \ref{eqn:lumi5}:
\begin{equation}
  \sigma_{\text{pixel}} = \left<N_{\text{cluster}}\right> f \mathcal{L}^{-1}
  \label{eqn:lumi5}
\end{equation}

The total systematic uncertainty on the luminosity measurement is 2.2\%.
The dominant contributions to this uncertainty are ``afterglow'' (energy
from late-arriving particles and from activated detector material)
and variations between Van der Meer scans.
