% Preamble is a separate file not to clog the main thesis file


% These instructions in a separate file because they will change often.
% This way we do not need to commit the changes every time we switch between draft and final layout.

\newcommand{\draftfinal}{%
  %draft%
  final% 
}
 % put these instructions in a separate file because they will change often

\documentclass[12pt,oneside,\draftfinal]{report}
% \documentclass[12pt,oneside,draft]{report}
\pagestyle{headings}

% Packages
\usepackage[T1]{fontenc}
\usepackage{setspace}
\usepackage{epsfig}
\usepackage{colordvi}
\usepackage{fancybox}
\usepackage{multicol}
\usepackage{multirow}
\usepackage{rotating}
\usepackage{dcolumn}
\usepackage{amsmath, amsthm, amssymb}
\usepackage{bm}
\usepackage{verbatim}
\usepackage{booktabs}
\usepackage{color}
\usepackage[usenames,dvipsnames,svgnames,table]{xcolor}
\usepackage{nth}
\usepackage{subcaption}
\usepackage[Glenn]{fncychap}

\doublespacing
\setlength{\topmargin}{-.5in}
\setlength{\textheight}{9in}
\setlength{\textwidth}{6.5in}
\setlength{\oddsidemargin}{2mm}
\setlength{\evensidemargin}{2mm}

% These are the margins that they require for the printed copy
% \setlength{\textwidth}{6.0in}
% \setlength{\oddsidemargin}{.4375in}
% \setlength{\evensidemargin}{.4375in}


%%%
\definecolor{darkblue}{rgb}{0,0,0.5} 
\usepackage
    [colorlinks=true,
      urlcolor=darkblue,
      anchorcolor=darkblue,
      linkcolor=darkblue,
      citecolor=darkblue,
      pdfauthor={Edmund A. Berry}
      pdfkeywords={CERN,CMS,Leptoquark},
      pdftitle={Edmund A. Berry, PhD Thesis},
      pdfsubject={Edmund A. Berry, PhD Thesis, Princeton University}]
    {hyperref}

%%% the follow sets the format for the header and footer
\usepackage{fancyhdr}
\addtolength{\headheight}{2.5pt}
\pagestyle{fancy} 
\fancyhead{}
\lhead{\leftmark}

% microtype improves the line filling
\usepackage{microtype}

\usepackage{graphicx}
% add here the location(s) of the figures, so that you don't need to add the path for each of them
% \graphicspath{{intro/fig/}{hadron/fig/}{event/fig/}{samples/fig/}{analysis/fig/}{limit/fig/}{conclusion/fig/}}
% do not need to specify the graph extensions (actually breaks pdflatex from picking up the pdf version of the file)
%\DeclareGraphicsExtensions{.eps,.ps,.eps.gz,.ps.gz,.pdf,{}}
\DeclareGraphicsRule{*}{mps}{*}{} 
\usepackage{sidecap}
\usepackage{wrapfig} % used to wrap text around figures
\usepackage{overpic} % used to annotate graphics (more intuitive units than lpic)

% -lpic- \usepackage{lpic} % used to annotate graphics
% -lpic- \newcounter{bean}
% -lpic- \setlength{\lpunitlength}{\textwidth}
% -lpic- \newcommand{\href}[2]{{#2}}

\usepackage{url}
% natbib handles the bibliography (formatting, ordering etc.)
\usepackage[square,sort&compress,numbers]{natbib}
%\usepackage{bibentry}   % for permissions
%%choose cite formatting:
%\renewcommand*{\citenumfont}[1]{\liningfigures{#1}} % oldstyle looks wrong to me in brackets.
%\citestyle{plain}           % numbers in square brackets
%\citestyle{nature}         % superscript.
%choose a number formatting for the biblist:
%\renewcommand*{\bibnumfmt}[1]{\eqparbox[t]{bblnm}{\hfill#1.}}
%

\usepackage[squaren]{SIunits}  % SIunits (squaren is to avoid conflicts with the ams square)
\usepackage{hepunits}  % hep units
\usepackage{hepnames}  % hep particles
\usepackage{cancel}    % to do the slashed (Feynman) symbols

% This package provides 'smart references'
% For example, while \ref{eq1} gives '(1)', \cref{eq1} gives 'eq.~(1)'. and \Cref{eq1} gives 'Eq.~(1)'
%\usepackage[english]{cleveref} % after hyperref
\usepackage{cleveref} % after hyperref
% the commands below are just to use 'figure' instead of the default 'fig.'
\crefname{equation}{}{equations}
\crefname{figure}{figure}{figures}
\crefformat{equation}{(#2#1#3)}
\Crefformat{equation}{Equation~(#2#1#3)}
\crefformat{appendix}{#2appendix~#1#3}
\Crefformat{appendix}{#2Appendix~#1#3}
\crefformat{chapter}{#2chapter~#1#3}
\Crefformat{chapter}{#2Chapter~#1#3}
\crefformat{section}{#2section~#1#3}
\Crefformat{section}{#2Section~#1#3}
\crefformat{figure}{#2figure~#1#3}
\Crefformat{figure}{#2Figure~#1#3}
\crefformat{table}{#2table~#1#3}
\Crefformat{table}{#2Table~#1#3}

% This package is to draw things in latex
\usepackage{tikz}
\usetikzlibrary{decorations.pathreplacing,fit}

% This package allows to add todo notes in the margins
\usepackage{todonotes}

%%% for list of abbreviations
% One day you might want to try acronym (which can produce clickable items) instead of nomencl:
% http://www.ctan.org/tex-archive/macros/latex/contrib/acronym/
\usepackage{nomencl}
\let\abbrev\nomenclature
\renewcommand{\nomname}{List of Abbreviations}
\setlength{\nomlabelwidth}{0.1\hsize}
%\renewcommand{\nomlabel}[1]{#1 \dotfill}
\setlength{\nomitemsep}{-\parsep}
\makenomenclature

%%% Set tick marks for lists:
\renewcommand{\labelitemi}{$\bullet$}
\renewcommand{\labelitemii}{$\circ$}



\title{Search for the Pair Production of \\ First Generation Scalar Leptoquarks \\ with the CMS Detector}

\author{Edmund A. Berry}
\date{
\vfill
{\normalsize A DISSERTATION PRESENTED TO THE \\
FACULTY OF PRINCETON UNIVERSITY \\
IN CANDIDACY FOR THE DEGREE OF \\
DOCTOR OF PHILOSOPHY \\
\vfill
RECOMMENDED FOR ACCEPTANCE \\
BY THE DEPARTMENT OF PHYSICS \\}
Advisor:  Christopher G. Tully \\
\vfill
April 2014
}


%%% input custom definitions here

\newcommand{\eejj}{$eejj$}
\newcommand{\enujj}{$e\nu jj$}
\newcommand{\nunujj}{$\nu\nu jj$}
\newcommand{\mumujj}{$\mu\mu jj$}
\newcommand{\munujj}{$\mu\nu jj$}
\newcommand{\emujj}{$e\mu jj$}
\newcommand{\zjets}{$\text{Z}^{0}$+jets}
\newcommand{\wjets}{$\text{W}$+jets}
\newcommand{\gjets}{$\gamma$+jets}
\newcommand{\ttbar}{$t\overline{t}$}

\newcommand{\dEtaIn} {$\Delta\eta_{\text{in}}$}
\newcommand{\dPhiIn} {$\Delta\phi_{\text{in}}$}
\newcommand{\HoE} {$H/E$}
\newcommand{\SigmaiEtaiEta} {$\sigma_{{\rm i}\eta{\text{i}}\eta}$}
\newcommand{\ETwoEFive} {$E^{2\times5}/E^{5\times5}$}
\newcommand{\EOneEFive} {$E^{1\times5}/E^{5\times5}$}
\newcommand{\EMIso}{$\text{EM}_{\text{Iso}}$}
\newcommand{\HADIso}{$\text{HAD}_{\text{Iso}}$}
\newcommand{\HADIsoOne}{$\text{HAD}^{\text{layer1}}_{\text{Iso}}$}
\newcommand{\HADIsoTwo}{$\text{HAD}^{\text{layer2}}_{\text{Iso}}$}
\newcommand{\TRKIso}{$\text{TRK}_{\text{Iso}}$}
\newcommand{\akt}{anti-$k_{\text{T}}$}

\newcommand{\pt}{\ensuremath{p_{\text{T}}}}
\newcommand{\ptvec}{\ensuremath{\vec{p}_{\text{T}}}}
\newcommand{\et}{\ensuremath{E_{\text{T}}}}
\newcommand{\Et}{\ensuremath{E_{\text{T}}}}
\newcommand{\st}{\ensuremath{S_{\text{T}}}}
\newcommand{\mee}{\ensuremath{m_{\text{ee}}}}
\newcommand{\mej}{\ensuremath{m_{\text{ej}}}}
\newcommand{\mejmin}{\ensuremath{m_{\text{ej}}^{\text{min}}}}
\newcommand{\mejavg}{\ensuremath{m_{\text{ej}}^{\text{avg}}}}
\newcommand{\metvec}{\ensuremath{\not\!\!{\vec{E}_{\text{T}}}}}
\newcommand{\met}{\ensuremath{\not\!\!{E_{\text{T}}}}}
\newcommand{\MET}{\ensuremath{\not\!\!{E_{\text{T}}}}}
\newcommand{\mt}{\ensuremath{m_{\text{T, e}\nu}}}
\newcommand{\mtjnu}{\ensuremath{m_{\text{T, j}\nu}}}
\newcommand{\Lint}{$\mathcal{L}_{int}$}
\newcommand{\minptmet}{$\mbox{min}(p^\text{e}_{\rm T},\mbox{\MET})$}
\newcommand{\dphimetele}{$\Delta\phi(\mbox{\MET},\text{e})$}
\newcommand{\dphimetjetone}{$\Delta\phi(\mbox{\MET},\mbox{j1})$}
\newcommand{\dphimetjettwo}{$\Delta\phi(\mbox{\MET},\mbox{j2})$}
\newcommand{\dRejone}{$\Delta R(e,\mbox{j1})$}
\newcommand{\dRejtwo}{$\Delta R(e,\mbox{j2})$}
\newcommand{\dRejets}{$\mbox{min}\Delta R(\text{e},\mbox{jets})$}
\newcommand{\MLQ}{$M_{\rm LQ}$}

\newcommand{\doublet}[2]{
  \left(\!
    \begin{array}{c}
      {#1} \\
      {#2}
    \end{array}
    \!\right) 
}

\def \lumi {4.95~fb$^{-1}$\xspace} 
\def \lumiPhoton {4.95~fb$^{-1}$\xspace}  %correct value: 4.656
\def \lumiSingleElePlusEleHad {4.95~fb$^{-1}$\xspace} %correct value: 4.649
\def  \lumiMuEG {4.95~fb$^{-1}$\xspace} %correct value: 4.667
\def \JSON {\tiny /afs/cern.ch/cms/CAF/CMSCOMM/COMM\_DQM/certification/Collisions11/7TeV/Prompt/Cert\_160404-180252\_7TeV\_PromptReco\_Collisions11\_JSON.txt \\ /afs/cern.ch/cms/CAF/CMSCOMM/COMM\_DQM/certification/Collisions11/7TeV/Reprocessing/Cert\_160404-163869\_7TeV\_May10ReReco\_Collisions11\_JSON\_v3.txt \\ /afs/cern.ch/cms/CAF/CMSCOMM/COMM\_DQM/certification/Collisions11/7TeV/Reprocessing/Cert\_170249-172619\_7TeV\_ReReco5Aug\_Collisions11\_JSON\_v3.txt \xspace}
\def \Npileup {9.5\xspace} %% : average number of pile-up interactions for the sample considered
\def \NpileupA {6.5\xspace} %% : average number of pile-up interactions for the sample considered
\def \NpileupB {12\xspace} %% : average number of pile-up interactions for the sample considered
\def \ContaminationAtZpeak {$\approx 5\%$\xspace} %% : contamination of other than Z backgrounds at the Z peak
\def \ContaminationAtWpeak {$\approx 40\%$\xspace} %% : contamination of other than W backgrounds at the W peak
\def \ContaminationAtTTBARenujj {$\approx 30\%$\xspace} %% : 
\def \ContaminationAtTTBARemujj {$\approx 2\%$\xspace} %% : 
\def \ContaminationAtQCDForClosureTestLooseLoose {$\approx 5\%$\xspace} %% : 
\def \ContaminationAtQCDForClosureTestLooseTight {$\approx 30\%$\xspace} %% : 
\def \ratioEmuRecoEff{$ 0.98 \pm 0.02 \mbox{ (stat.)} \pm 0.02 \mbox{ (syst.)}$\xspace} %% : use updated rescaling factor
\def \rescaleFactorTTBAReejj{$ 0.49 \pm 0.01 \mbox{ (stat.)} \pm 0.01 \mbox{ (syst.)}$\xspace} %% : use updated rescaling factor
\def \rescaleFactorTTBARenujj{$0.72 \pm 0.06 \mbox{ (stat.)} \pm 0.04 \mbox{ (syst.)}$}
\def \rescaleFactorWSherpa{$1.26 \pm 0.05 \mbox{ (stat.)} \pm 0.03 \mbox{ (syst.)}$}
\def \rescaleFactorZSherpa{$1.27 \pm 0.02 \mbox{ (stat.)} \pm 0.001 \mbox{ (syst.)} $}
\def \QCDcontributionINeejj {$\approx 1\%$\xspace} %% : 
\def \QCDcontributionINenujj {$\approx 8\%$\xspace} %% : 
\def \OtherBackgroundContributionINeejj {$10\%$\xspace} %% : 
\def \OtherBackgroundContributionINenujj {$20\%$\xspace} %% : 
\def \NeventsLTJpred{$10730 \pm 941$\xspace} %% : 
\def \NeventsLTJactual{$14790 \pm 465$\xspace} %% : 
\def \NeventsLTJpredSTCUT{$290 \pm 22$\xspace} %% : 
\def \NeventsLTJactualSTCUT{$232 \pm 40$\xspace} %% : 
\def \RatioLTJpredVSactual{$0.73 \pm 0.06$\xspace} %% : 
\def \RatioLTJpredVSactualSTCUT{$1.25 \pm 0.23$\xspace} %% : 
\def \QCDUncertEnujj{25\%\xspace} %%: 
\def \QCDUncertEEjj{50\%\xspace} %% :
\def \ObservedLimitBetaOneeejj{830\xspace} %% :
\def \ObservedLimitBetaHalfenujj{640\xspace} %% :
\def \ObservedLimitBetaHalfenujjCombined{640\xspace}
\def \ObservedLimitBetaHalfenujjOnlyMCStat{642\xspace}
\def \ExpectedLimitBetaOneeejj{790\xspace} %% :
\def \ExpectedLimitBetaHalfenujj{640\xspace} %% :
\def \ExpectedLimitBetaHalfenujjCombined{680\xspace}

\def \EESEB{1\%}
\def \EESEE{3\%}
\def \EEREB{1\%}
\def \EEREE{3\%}
\def \JES{4\% \xspace}
\def \BkgShapeUncerZJetsEEJJ{15\%\xspace}
\def \BkgShapeUncerWJetsENUJJ{20\%\xspace}
\def \BkgShapeUncerTtbarENUJJ{10\%\xspace}

