\section{Monte Carlo samples}
\label{sec:mc-samples}

The collision data described in Section \ref{sec:data-samples}
are compared to samples of Monte Carlo (MC) generated events.
Different processes were modeled using MC generators, as detailed below.
For the generation of all the MC samples, the {\tt CTEQ6L1}~\cite{cteq}
parton distribution functions (PDFs) were used.
The response of the detector was simulated using
{\sc Geant4} as described in Section \ref{sec:reco-gen}. 
The detector geometry description included
realistic subsystem conditions such as defunct and noisy channels.

The signal pair-production scalar first generation leptoquark MC samples were generated using 
{\sc Pythia}~\cite{pythia}, version 6.422, tune Z2.
Two different sets of leptoquark samples are produced: 
one where both leptoquarks decay to electrons and up quarks (for the \eejj~channel), and 
one where one leptoquark decays to an electron and an up quark and the
other leptoquark decays to an electron neutrino and a down quark (for the \enujj~channel).
In both cases, the leptoquarks are produced with a coupling equal to the
electromagnetic coupling, $\lambda = \lambda_{\text{em}} = 0.3$.
The leptoquarks modeled by the MC have an electric charge of $\pm 1/3$, corresponding to leptoquarks
of type $S_{0,L}$ and $S_{1,L}$ as described by the Lagrangian in 
Equation \ref{eqn:brw-lagrangian-f2} and shown in the
top row of Table \ref{tab:lq-classification}.
However, since neither the search in the \eejj~channel nor the search in the \enujj~channel
are sensitive to leptoquark electric charge, hypercharge,
or weak isospin, the leptoquarks modeled by the MC are
representative of all of the scalar leptoquarks described
by the mBRW model.
11 samples were generated for the \eejj~channel, and 10 samples were generated for the \enujj~channel
over different ranges of leptoquark masses.
The range of leptoquark masses extends from 250 GeV to 900 for the \eejj~channel and
from 250 to 850 for the \enujj~channel.  A broader mass range is needed for the \eejj~channel
because it has lower background and greater sensitivity than the \enujj~channel.
Table \ref{tab:lq-samples} shows the leptoquark mass value,
final state, NLO cross section, number of generated events, and equivalent luminosity
for each signal leptoquark sample.

\begin{table}
  \centering
  \begin{tabular}{c|c|ccc}
    M(LQ) $[\text{GeV}]$ & Final state & $\sigma \times \text{BR}$ (NLO) $[\text{pb}]$ & 
    Events generated & Equivalent $\mathcal{L}_{\text{int}}$ [$\text{fb}^{-1}$] \\
    \hline\hline
    250 & \eejj & 3.47 & 64656 & 18.6 \\ 
    350 & \eejj & 0.477 & 67528 & 141.6 \\ 
    400 & \eejj & 0.205 & 52450 & 255.9 \\ 
    450 & \eejj & $0.948 \cdot 10^{ -1 }$ & 52382 & 552.6 \\ 
    500 & \eejj & $0.463 \cdot 10^{ -1 }$ & 61888 & 1336.7 \\ 
    550 & \eejj & $0.236 \cdot 10^{ -1 }$ & 61910 & 2623.3 \\ 
    600 & \eejj & $0.124 \cdot 10^{ -1 }$ & 49665 & 4005.2 \\ 
    650 & \eejj & $0.676 \cdot 10^{ -2 }$ & 63664 & 9417.8 \\ 
    750 & \eejj & $0.214 \cdot 10^{ -2 }$ & 51745 & 24179.9 \\ 
    850 & \eejj & $0.732 \cdot 10^{ -3 }$ & 52432 & 71628.4 \\ 
    900 & \eejj & $0.436 \cdot 10^{ -3 }$ & 59779 & 137107.8 \\ 
    \hline
    250 & \enujj & 1.74 & 60119 & 34.7 \\ 
    350 & \enujj & 0.238 & 58891 & 246.9 \\ 
    400 & \enujj & 0.102 & 68538 & 668.7 \\ 
    450 & \enujj & $0.474 \cdot 10^{ -1 }$ & 59746 & 1260.5 \\ 
    500 & \enujj & $0.232 \cdot 10^{ -1 }$ & 52636 & 2273.7 \\ 
    550 & \enujj & $0.118 \cdot 10^{ -1 }$ & 64246 & 5444.6 \\ 
    600 & \enujj & $0.620 \cdot 10^{ -2 }$ & 61474 & 9915.2 \\ 
    650 & \enujj & $0.338 \cdot 10^{ -2 }$ & 57873 & 17122.2 \\ 
    750 & \enujj & $0.107 \cdot 10^{ -2 }$ & 58450 & 54626.2 \\ 
    850 & \enujj & $0.366 \cdot 10^{ -3 }$ & 54181 & 148035.5 \\ 
  \end{tabular}
  \caption{
    Leptoquark masses, final states, NLO cross sections, number of events generated, and equivalent luminosities 
    for pair-production scalar leptoquark MC samples.  All values correspond to 
    $pp$ collisions at $\sqrt{s} = 7$~TeV (the LHC collision energy in 2010 and 2011).
    Cross sections are given in units of pb ($1 \text{ b}= 10^{-28} \text{m}^2 = 10^{-24} \text{cm}^2$).
    In all cases, the renormalization and factorization scale is set to be equal to the 
    leptoquark mass. 
  }
  \label{tab:lq-samples}
\end{table}

MC samples were also generated to model the Standard Model background.
\wjets~and \zjets~events were generated using {\sc Sherpa}~\cite{sherpa}.
The \ttbar~and \gjets~events were generated using {\sc MadGraph}~\cite{madgraph-1,madgraph-2}.
The single top events were generated using {\sc POWHEG}~\cite{powheg-st,powheg-w,powheg-1,powheg-2,powheg-3}.
The diboson (WW, ZZ, WZ) events were generated using {\sc Pythia}.
For the {\sc MadGraph} and {\sc POWHEG} samples, parton showering and hadronization were performed with {\sc Pythia}.
The QCD multijet background is estimated from data, as described in Section~\ref{sec:qcd}.

The total {\sc Sherpa} cross sections for the \wjets~and \zjets~samples
were rescaled to inclusive next-to-next-to leading order (NNLO) values
of 31314~pb (for \wjets) and 3048~pb (for $\text{Z}/\gamma \rightarrow \ell\ell$ with
$\text{M}_{\ell\ell}>50$~GeV).  The {\sc Sherpa} cross sections were
calculated using {\sc FEWZ}~\cite{fewz}.
The {\sc MadGraph} \ttbar~sample was normalized to an 
inclusive next-to-next-to-leading-logarithm (NNLL)
\nomenclature{NNLL}{Next-to-next-to leading logarithm}
cross section of 163~pb \cite{ttbar-xsection}.
The \gjets~samples were normalized to LO cross sections calculated with {\sc MadGraph}.
The single top samples were generated via three distinct
channels: $s$-channel, $t$-channel, and $tW$-channel.
An NNLL cross section of 3.6~pb was calculated for the s-channel~\cite{singletop-xsection}.
NLO cross sections of 54.1~pb and 14.9~pb were calculated for the t-channel and tW-channel, respectively,
using {\sc MCFM}~\cite{mcfm}.
The WW, WZ, and ZZ samples were normalized to NLO cross sections of 47, 18.2, and 7.4~pb, 
respectively, calculated with {\sc MCFM}. 

