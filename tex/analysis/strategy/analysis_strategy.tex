\chapter{Analysis: Strategy}
\label{ch:analysis-strategy}

Leptoquarks present a very attractive target for searches at the LHC. 
As discussed in Section \ref{sec:lq-ppcollisions}, 
the pair production cross section of scalar leptoquarks at a $pp$ 
collider like the LHC depends only on the leptoquark mass.
As shown in Table \ref{tab:lq-xsection}, this 
cross section is relatively high: mBRW leptoquarks with a mass of 1 TeV 
have a pair production cross section of approximately 
$1.6 \cdot 10^{-4}$~pb in $pp$ collisions at $\sqrt{s} = 7$~TeV.
In addition, if $\beta = \text{BR}(\text{LQ}\rightarrow\ell^{\pm}q)$
is not zero, first generation leptoquarks produced at the LHC would decay to
isolated, energetic electrons, which could be easily triggered upon 
(Section \ref{sec:trigger}) and reconstructed (Section \ref{sec:reco-electron}) using the CMS detector 
with relatively little background from Standard Model processes.

This thesis takes $\beta$ as a free parameter, which means that $\beta$ is allowed
to take any value between 0 and 1, rather than being fixed to a value of $0$,
$1/2$, or $1$ as discussed in Section \ref{sec:brw-model}.
This leads to a search for first generation leptoquarks in two channels.
In the first channel, $\beta$ is taken to be equal to 1, and 
the decay of pair-produced leptoquarks to two electrons and two jets
($\text{LQ}\overline{\text{LQ}}\rightarrow eejj$) is considered.
In the second channel, $\beta$ is taken to be equal to $1/2$, and 
the decay of pair-produced leptoquarks to one electron, one neutrino, and two jets 
($\text{LQ}\overline{\text{LQ}}\rightarrow e\nu jj$) is considered.
As discussed in Section \ref{sec:lq-decays}, a third channel 
where $\beta$ is taken to be zero and pair-produced leptoquarks decay to two 
neutrinos and two jets 
($\text{LQ}\overline{\text{LQ}}\rightarrow \nu\nu jj$) is possible.  
While the $\beta = 0$ channel is not 
covered by this analysis, other analyses at CMS are sensitive to decays 
of massive scalar particles to a final state of \met~and two or more energetic jets \cite{RA1susy-2011}.
In the rest of this paper, the search for the pair production of scalar 
leptoquarks where $\beta = 1$ will be referred to as the ``\eejj~channel'',
and the search where $\beta = 1/2$ will be referred to as the ``\enujj~channel.''
Results from these two searches may be combined to search for leptoquarks with 
values of $\beta$ anywhere between 0 and 1  with varying sensitivity.

The basic strategy for the \eejj~channel is to trigger on events with two electrons
and then to apply an offline selection requiring exactly two electrons and at least two jets.  
Neither the \eejj~channel nor the \enujj~channel
requires exactly two jets, since additional jets are frequently produced in $pp$ collisions at the LHC
by initial state and final state gluon radiation (ISR and FSR).  
\nomenclature{ISR}{Initial state radiation}
\nomenclature{FSR}{Final state radiation}
Both analyses, however,
only use the two leading jets in \pt~to reconstruct leptoquark candidates.
Several Standard Model processes exist that produce a final state of two electrons and at least two jets.
The most significant of these background processes for the \eejj~channel are \zjets~and \ttbar.
Other background processes in the \eejj~channel include 
dibosons (WW, WZ, and ZZ), single top,
\gjets~(where the photon converts to two electrons),
\wjets~(where at least one jet is misidentified as an electron)
and QCD multijets (where at least two jets are misidentified as electrons).
Several variables serve as descriminants to separate leptoquark
signal from these Standard Model background processes.
Both the \eejj~channel and the \enujj~channel define an
\st~variable as the scalar sum of the most energetic objects in the
final state.  In the case of the \eejj~channel, \st~is defined
using Equation \ref{eqn:st-eejj}:
\begin{equation}
  \st = \pt(e_1) + \pt(e_2) + \pt(j_1) + \pt(j_2) 
  \label{eqn:st-eejj}
\end{equation}
where 
$\pt(e_1)$ is the scalar value of the \pt~of the leading electron in \pt,
$\pt(e_2)$ is the scalar value of the \pt~of the second leading electron in \pt,
$\pt(j_1)$ is the scalar value of the \pt~of the leading jet in \pt, and
$\pt(j_2)$ is the scalar value of the \pt~of the second leading jet in \pt.
The average value of \st~is significantly larger for signal events, due to the 
energetic decay products produced by a heavy boson like a leptoquark.
Another effective discriminant in the \eejj~channel is the invariant mass
of the electron pair in the event (\mee).  For \zjets~events
(the Standard Model background in the \eejj~channel with the highest cross section), the \mee~distribution
forms a peak at the $Z$ mass of 91.2 GeV, while it is significantly higher
on average for leptoquarks.  The final discriminant used in the \eejj~channel 
comes from reconstructing leptoquark candidates.  
In the \eejj~channel, electrons are paired with 
jets to reconstruct leptoquark candidates.  
There are two possible ways to combine two electrons ($e_1$ and $e_2$) and two jets ($j_1$ and $j_2$)
to form two LQ candidates: ($e_1$--$j_1$,$e_2$--$j_2$)  or ($e_1$--$j_2$, $e_2$--$j_1$).
The pairing that minimizes the difference
between the invariant masses of the resulting reconstructed leptoquark candidates is chosen to form leptoquark candidates.
The mass of the least massive reconstructed leptoquark candidate (\mejmin) is used as the final
discriminant in the \eejj~channel.

Similarly, the basic strategy for the \enujj~channel is to trigger on events with exactly one
electron, large \met, and at least two jets and then to apply a tighter offline selection.
In the case of the \enujj~channel, the most significant background processes are 
\wjets~and \ttbar.
Other background processes in the \enujj~channel include dibosons, single top,
\zjets~(where one electron is not reconstructed)
and QCD multijets (where one jet is misidentified as an electron).
A similar \st~variable is defined for the \enujj~channel using Equation \ref{eqn:st-enujj}:
\begin{equation}
  \st = \pt(e_1) + \met + \pt(j_1) + \pt(j_2) 
  \label{eqn:st-enujj}
\end{equation}
where $\pt(e_1)$, $\pt(j_1)$, and $\pt(j_2)$ have the same definition
as in Equation \ref{eqn:st-eejj}, and \met~represents the energy imbalance
in the transverse plane as reconstructed by the PF algorithm.
As in the \eejj~channel, the average value of \st~is significantly larger for signal events in the \enujj~channel, due to the 
energetic decay products produced by a heavy boson like a leptoquark.
In addition to the \st~variable, another effective discriminant in the \enujj~channel
is the transverse mass of the electron and \met~in the event (\mt).
\mt~is defined by Equation \ref{eqn:mt}:
\begin{equation}
  \mt = \sqrt{2 \cdot \pt(e_1) \cdot \left[1 - \text{cos}(\Delta\phi(e_1,\met))\right]}
  \label{eqn:mt}
\end{equation}
where $\pt(e_1)$ and \met~have the same definition as in Equation \ref{eqn:st-enujj},
and $\Delta\phi(e_1,\met)$ is the opening angle in $\phi$ between the leading electron
in \pt~and the \met.  For \wjets~events (the Standard Model background in the 
\enujj~channel with the highest cross section), the \mt~distribution forms a Jacobian
peak close to W mass of 80.4 GeV, while it is significantly higher on average for 
leptoquarks.
Another effective discriminant between leptoquarks and \ttbar~and \wjets~background
events is \met~itself.  \met~is expected to be significantly higher on average
for leptoquark events than for \ttbar~and \wjets~background events, due to the energetic
neutrino produced by the leptoquark decay.
The final discriminant used for the \enujj~channel comes from reconstructing leptoquark candidates.  
Similarly to the \eejj~channel, there are two possible ways to combine one electron ($e_1$, one neutrino (i.e. \MET) and two jets ($j_1$ and $j_2$)
to form two leptoquark candidates: ($e$--$j_1$, \met--$j_2$) or ($e$--$j_2$,\met--$j_1$). 
The pairing that minimizes the difference
between the transverse masses of the resulting reconstructed leptoquark candidates is chosen to form leptoquark candidates.
The invariant mass of the electron-jet pairing (\mej) is used as the final discriminant in the
\enujj~channel.

In both the \eejj~channel and the \enujj~channel, 
a preselection is defined that is dominated by Standard Model
background, in order to study the modeling of the various background
predictions.  Once the backgrounds are well understood,
a series of final selections is applied using the discriminants for each
channel: 
\st, \mee, and \mejmin~for the \eejj~channel and 
\st, \mt, \met, and \mej~for the \enujj~channel.
Each final selection is optimized for a given hypothesis for the leptoquark invariant mass,
ranging between 250 and 900 GeV.
Further details on the datasets used by this analysis, preselection, final selection optimization,
and background modeling are given in the following chapters.
