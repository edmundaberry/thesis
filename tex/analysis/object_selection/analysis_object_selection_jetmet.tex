\section{Jets}

The jets used in this analysis are reconstructed
by running the anti-$k_{\text{T}}$ jet algorithm with a 
radius parameter of 0.5 over PF candidates, as described in Section
\ref{sec:jetmet}.  On top of the selection applied during 
reconstruction and detector electronics noise cleaning applied during jet reconstruction, 
a further jet identification selection is applied to reduce contamination from detector electronics noise or
single particles (electrons, for example) that are reconstructed as jets.
This additional identification selection is particular to PFJets 
and is referred to as ``loose jet ID''.

The loose jet ID is a cut-based ID, and the contributing variables are 
described below.
The numerical requirements for each of these variables are shown
in Table \ref{tab:jetid}.

\begin{table}
  \centering
  \begin{tabular}{c|c}
    Variable & Criterion \\
    \hline\hline
    \multicolumn{2}{c}{For all jets} \\
    \hline
    Neutral hadronic fraction & $<0.99$ \\
    Neutral electromagnetic fraction & $<0.99$ \\
    Number of constituents & $> 1$ \\
    \hline
    \multicolumn{2}{c}{For jets with $|\eta| < 2.4$} \\
    \hline
    Charged hadronic fraction & $>0$ \\
    Charged electromagnetic fraction & $<0.99$ \\
    Charged multiplicity & $>0$ \\
  \end{tabular}
  \caption{The loose jet ID selection criteria for PFJets}
  \label{tab:jetid}
\end{table}

\begin{itemize}
  \item {\bf Neutral hadron fraction:} fraction of the total jet energy associated with hadronic energy deposits that are not linked to tracks
  \item {\bf Neutral electromagnetic fraction:} fraction of the total jet energy associated with electromagnetic energy deposits that are not linked to tracks
  \item {\bf Number of constituents:} number of PF candidates included in the jet
  \item {\bf Charged hadron fraction:} fraction of the total jet energy associated with hadronic energy deposits that are linked to tracks
  \item {\bf Charged electromagnetic fraction:} fraction of the total jet energy associated with electromagnetic energy deposits that are linked to tracks
  \item {\bf Charged multiplicity:} number of charged PF candidates included in the jet
\end{itemize}

Only jets with $\pt > 30$ GeV are considered
for the \eejj~analysis, and only jets with $\pt > 40$ GeV
are considered for the \enujj~analysis.  In both analyses,
only jets with $|\eta| < 2.4$ are considered.
Finally, to address the double-counting of well identified and isolated 
electrons and muons reconstructed also as jets, for each selected electron 
or muon the closest selected jet within 
$\Delta R=0.3$\footnote{$\Delta R$ is defined as $\sqrt{\Delta\phi^2+\Delta\eta^2}$}
is removed from the jet collection.  In this context,
``selected'' electrons, muons, and jets are required to pass the \pt~cut of 
the event pre-selection described in the next sections, 
and the ID/isolation and noise cleaning requirements.

