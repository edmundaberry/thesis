\section{Muons}
\label{sec:id-muon}


The muon candidates used in this analysis are taken from
global muons, which are reconstructed according to the process
described in Section \ref{sec:reco-muon}.
On top of the selection applied during reconstruction,
a further muon identification selection is applied to reduce contamination from various backgrounds.
This additional identification selection is known as the ``tight'' muon ID, and it 
was developed by the muon physics object group within CMS.  
Significant backgrounds include muons produced within jets
and hadrons that are incorrectly identified as muons.  
These misreconstructed objects are refered to as ``fake muons'', and two important sources of them are
hadrons decaying to muons while passing through the detector 
(e.g. $\pi^{\pm} \rightarrow \mu^{\pm} \nu_{\mu}$, ``decay in flight'')
and hadrons that pass through the entire detector and interact with the 
muon subsystem without showering (``punch through'').
The tight muon ID has been studied in detail
\cite{muon-1}, and it is used by many CMS analyses that include muons in their final
stat, including the second generation leptoquark analysis \cite{pair-lq-CMS}.

Like the HEEP ID for electrons,
the tight muon ID is a cut-based ID, and the variables contributing to it are described
below. The numerical requirements for each of these variables are shown
in Table \ref{tab:muon-tight}.

\begin{table}
  \centering
  \begin{tabular}{c|c}
    Variable & Criterion \\
    \hline\hline
    Pixel hits & $>0$ \\
    Tracker hits & $>10$ \\
    Muon chamber hits & $>0$ \\
    Muon stations & $>1$ \\
    $|d_{xy}|$ & $<2 \text{ mm}$ \\
    $\chi^2/\text{ndof}$ & $<10$ \\
    Tracker isolation & $< 3$ GeV \\
  \end{tabular}
  \caption{The ``tight'' selection criteria for muon ID and isolation.}
  \label{tab:muon-tight}
\end{table}

\begin{itemize}
  \item {\bf Muon reconstruction algorithm.} Several muon reconstruction algorithms 
    are used by various analyses at CMS.  In this analysis, the muon must be a global muon.
  \item {\bf Pixel hits:} number of hits in the pixel detector.  This value must be large
    in order to suppress muons from decays in flight.
  \item {\bf Tracker hits:} number of hits in the inner tracker system (including the pixel detector) included in the global
    muon fit.  This number must be large in order to guarantee a good \pt~measurement and to 
    suppress muons from decays in flight.
  \item {\bf Muon chamber hits:} number of hits in the muon subsystem included in the global
    muon fit.  This number must be large in order to suppress 
    fake muons from punch through and decays in flight.
  \item {\bf Muon stations:} number of muon stations with muon segments in them which contribute
    to the global muon fit.  This number must be large in order to suppress fake muons from punch through
    and accidental matches between unrelated tracks from the inner 
    tracker and segments from the muon system.  In addition, requiring at least two muon stations
    makes this selection consistent with the logic of the muon trigger.
  \item {\bf $\chi^2/\text{ndof}$:} an evaluation of the quality of the global muon fit.
    This value must be small in order to suppress fake muons from punch through and decays in flight.
  \item {\bf $d_{xy}$:} transverse impact parameter with respect to the primary vertex.  This
    number must be small in order to suppress cosmic muons and to 
    suppress fake muons from decays in flight.  The cut is loose enough to preserve efficiency for 
    muons from decays of long-lived hadrons containing $b$ and $c$ quarks.
  \item {\bf Tracker isolation:} the sum of the \pt~of the tracks within a cone
    of radius $\sqrt{\Delta\phi^2\times\Delta\eta^2} = 0.3$ around the muon track, excluding the muon 
    track itself.  This requirement is not part of the tight muon ID.  It is applied in addition
    to the tight muon ID in order to suppress muons that are contained within jets.
\end{itemize}

Additionally for this analysis, muon candidates must have $\pt > 30$ and at least one muon in the event
must be reconstructed in the HLT fiducial volume, $i.e.$ with $|\eta| < 2.1$.

