%
% Sketch of the SM particles and interactions
%
\begin{tikzpicture}
  [x=16cm,y=8cm,node distance=2cm,auto,
    every loop/.style={},
    % particle/.style={circle,draw=blue!50,fill=blue!20,thick}]
    particle/.style={circle,draw=black!50,fill=white!20,thick}]
  % Particles
  \node[particle] at (4cm,7cm) (L) {$\Plepton^{\pm}$};
  \node at (3cm,9cm) (leptons) {  % table with leptons
    \begin{tabular}{ccc}
      \multicolumn{3}{c}{Leptons (\Pneutrino, $\Plepton^{\pm}$)} \\
      \hline 
      \Pnue & \Pnum & \Pnut \\
      \Pe & \Pmu & \Ptau \\
    \end{tabular}
  };
  \node[right of=L] (dummyRL) {};
  \node[right of=dummyRL] (dummyRRL) {};
  \node[below left of=L] (dummyBL) {};
  \node[particle] (Q) [right of=dummyRRL]{\Pquark};
  \node at (10cm,9cm) (quarks) {  % table with quarks
    \begin{tabular}{ccc}
      \multicolumn{3}{c}{Quarks (\Pquark)} \\
      \hline
      \Pup & \Pcharm & \Ptop \\
      \Pdown & \Pstrange & \Pbottom \\
    \end{tabular}
  };
  \node[particle] (L2) [above left of=dummyBL]{\Pneutrino};
  \node[particle] (P) [below left of=dummyBL]{\Pphoton};
  \node[particle] (W) [below right of=dummyBL]{\PW};
  \node[particle] (Z) [right of=W]{\PZ};
  \node[right of=Z] (dummyRZ) {};
  \node[particle] (G) [right of=dummyRZ]{\Pgluon};
  \node[below of=P] (dummyBP) {};
  \node[particle] (H) [below right of=dummyBP]{\PH};

  % Interactions
  \path
  (L)  edge [bend right] (P)
  (L)  edge [bend right] (H)
  (L)  edge [bend left] (W)
  (L)  edge [bend left] (Z)
  (L2)  edge [bend left] (H)
  (L2)  edge [bend left] (W)
  (L2)  edge [bend left] (Z)
  % (L2)  edge [bend left] (L)
  (Q) edge [bend right=20] (P)
  (Q) edge [bend right=20] (W)
  (Q) edge [bend right=20] (Z)
  (Q) edge [bend left=20] (G)
  (Q) edge [bend left=20] (H)
  (P) edge [bend right] (W)
  (W) edge [loop above] (W)
  (W) edge [bend right] (Z)
  (W) edge [bend left] (H)
  (Z) edge [bend left=20] (H)
  (H) edge [loop below] (H)
  (G) edge [loop below] (G);

  % Draw curly braces using path decoration
  \draw [gray,decorate,decoration={brace,amplitude=5pt}]
  (-0.25cm,6.25cm) -- (-0.25cm,7.75cm)
  node [black,midway,below=0pt,xshift=-2pt,rotate=-90] {Spin 1/2};
  \draw [gray,decorate,decoration={brace,amplitude=5pt}]
  (-0.25cm,3.25cm) -- (-0.25cm,4.75cm)
  node [black,midway,below=0pt,xshift=-2pt,rotate=-90] {Spin 1};
  \draw [gray,decorate,decoration={brace,amplitude=5pt}]
  (-0.25cm,0.00cm) -- (-0.25cm,1.5cm)
  node [black,midway,below=0pt,xshift=-2pt,rotate=-90] {Spin 0};
  % Draw mass scales
  \draw (5.40cm,-0.30cm) rectangle (13.10cm,1.5cm); % draw the box
  \draw[->,xshift=5.5cm] (0cm,0.5cm) -- coordinate (x axis mid) (7.5cm,0.5cm);
  \node[below=0.10cm, xshift=3.25cm] at (x axis mid) {mass};
  \foreach \x/\text in {1/meV, 2/eV,3/keV,4/MeV,5/GeV,6/TeV }
      \draw [xshift=5.0cm,yshift=0.5cm](\x cm,1pt) -- (\x cm,-3pt)
      node[anchor=north] {\text};
  % \node[draw=blue,inner sep=0pt,thick,circle,fit=(x axis mid)] {}; % test fit feature box
  % should enter the mass values and actually compute the log to get x...
  \node [text height=10pt] at (5.0cm+1.5cm,0.7cm +2pt) (mnu) {\Pnue,\Pnum,\Pnut};
  \node [text height=10pt] at (5.0cm+3.5cm,0.7cm +2pt) (mel) {\Pe};
  \node [text height=10pt] at (5.0cm+4.5cm,0.7cm +2pt) (mmu) {\Pmu};
  \node [text height=10pt] at (5.0cm+5.05cm,0.7cm+2pt) (mtau) {\Ptau};
  \node [text height=10pt] at (5.0cm+3.8cm,0.7cm+12pt) (mud) {$\Pup\Pdown$};
  \node [text height=10pt] at (5.0cm+4.5cm,0.7cm+12pt) (ms) {\Pstrange};
  \node [text height=10pt] at (5.0cm+5.05cm,0.7cm+12pt) (mcb) {$\Pcharm\Pbottom$};
  \node [text height=10pt] at (5.0cm+5.75cm,0.7cm+12pt) (mt) {\Ptop};

\end{tikzpicture}

