\chapter{Conclusion and prospects}
\label{ch:conclusion}

This thesis has described a search for the
pair production of first generation scalar leptoquarks
in $\sqrt{s} = 7$ TeV $pp$ collisions 
using the CMS detector at the LHC.
The data used for this search was collected in 2011 and 
corresponds to an integrated
luminosity of \lumi.  For each leptoquark
final state and mass under consideration,
the number of events observed is in good agreement with 
the Standard Model prediction.  In light of this, a
95\% confidence level combined lower limit is set on the mass of a
first generation scalar leptoquark at 830 and 640 GeV for $\beta = 1$ and 0.5, respectively,
where $\beta$ is the branching fraction of the leptoquark to an electron
and a quark.  This represents a substantial improvement over the 2010 
interation of this analysis, which 
excluded the production of first generation scalar leptoquarks
with masses less than 384 and 340 GeV, when assuming $\beta = 1$ and 0.5, respectively \cite{lq1-paolo,lq1-dinko}.

This analysis is currently being updated for the 2012 CMS dataset, 
which consists of $\sqrt{s} = 8$ TeV $pp$ collisions and corresponds to 
an integrated luminosity of 19.6 $\text{fb}^{-1}$.
This significant improvement in collision energy and dataset size will extend 
the sensitivity to leptoquarks masses up to 1 TeV (750 GeV) in the \eejj~(\enujj)
channel.  Even higher collision energies, larger datasets, and stronger sensitivities 
await this analysis after the end of LS1 and the continuation of the LHC $pp$ physics program.

