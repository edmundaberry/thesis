\section{Electron reconstruction}
\label{sec:reco-electron}

Electrons in CMS are reconstructed using input from two detectors:
the inner tracker, which reconstructs electron tracks, and the ECAL,
into which electrons deposit most of their energy.
The primary challenge in reconstructing electrons stems from the fact
that electrons emit a large fraction of their total energy via 
Bremsstrahlung radiation while traversing the inner tracker.  
If an electron undergoes Bremsstrahlung radiation within the tracker,
the electron trajectory may be significantly altered.  These deviations
must be accounted for by both track and energy reconstruction algorithms.

Just as there are two complimentary subdetectors with which to measure electrons
in the CMS detector, CMS uses two complimentary algorithms to seed the 
track reconstruction process: tracker-driven seeding and ECAL-driven 
seeding.  It is possible for an electron to be reconstructed using both algorithms.
The tracker-driven seeding algorithm performs best for electrons that have 
low \pt~or are buried inside jets.
The ECAL-driven seeding algorithm is optimized for isolated electrons down to 
$\pt \simeq 5$~\GeV, which (in principle) includes electrons from leptoquark decays.
Electrons radiating within the tracker tend to deposit their energy within 
several ECAL crystals.  Approximately 94\% of the incident electron energy
is contained in $3\times3$~crystals, and 97\% of the incident electron
energy is contained within $5\times5$ crystals \cite{cms-tdr}.  In addition,
electron energy tends to be deposited in the ECAL in a narrow region in 
pseudorapidity and a longer region in $\phi$, due to the electron's trajectory 
within the magnetic field.  To capture and reconstruct as much of this energy 
as possible without using input from empty or unnecessary crystals, 
the electron reconstruction algorithm makes use of a ``supercluster'' pattern.
A supercluster is a collection of one or more clusters of energy deposits in 
the ECAL within a narrow region in pseudorapidity and a longer region in $\phi$.
The ECAL-driven seeding algorithm begins by considering superclusters with transverse energy
greater than 4~\GeV~and a ratio of hadronic energy from behind the supercluster
over the supercluster energy of $H/E < 0.15$.  
As a first filtering step, these superclusters are required to be matched to
track seeds, which are composed of pairs or triplets of hits in the inner
track layers.  The electron trajectory is reconstructed using a dedicated
modeling of the electron energy loss via Bremsstrahlung radiation
and fitted with a Gaussian Sum Filter (GSF)
\nomenclature{GSF}{Gaussian sum filter} \cite{gsf}.
The final electron energy measurement may be taken from the GSF track,
the supercluster energy, or some combination of the two
\cite{electron-1,electron-2,electron-3}.

The first filtering step is complimented by a preselection, which varies
depending on the seeding algorithm used.  In the case of the tracker-driven
seeding, the preselection is based on a multivariate analysis \cite{pf-1}.
In the case of the ECAL-driven seeding, the preselection is based on
matching the supercluster to the GSF track in $\eta$ and $\phi$ \cite{electron-1}.
