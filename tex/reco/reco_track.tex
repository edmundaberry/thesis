\section{Track reconstruction}
\label{sec:reco-track}

Reconstruted charged particle trajectories or ``tracks''
are formed using information from the pixel tracker
and the silicon strip tracker.  Because the track reconstruction
process begins before any primary decay vertices have been 
reconstructed, it is dependent on a precise estimate of the ``beamspot'',
i.e. the location of the interaction point in the transverse plane, 
which is calculated using a beamspot fitter \cite{beamspot-fitter}.

Once the beamspot position has been estimated, track reconstruction proceeds
using an iterative process, which is performed by the combinatorial
track finder (CTF)
\nomenclature{CTF}{Combinatorial track finder}
\cite{track-reco-2010,ctf-1,ctf-2}.
Each iteration begins reconstructing tracks using ``seeds'', 
which may be thought of as starting points for initial estimates of fully
reconstructed particle tracks.  
Seeds are either triplets of hits in the tracker or pairs of hits with 
the beamspot or a pixel vertex used as an additional constraint.  
These seeds yield a preliminary estimate of the particle track with an associated
uncertainty, and they are then propagated outward in a search for additional
hits to associate with the track.
Once found, compatible hits are added to the track, and the track's 
parameters and associated uncertainties are recalculated.
In each iteration, the outward search for hits continues until either the boundary of the tracker has been
reached or no additional hits can be found.  The search is then repeated by starting from the 
outermost hits and propagating them inward.  The final step of each iteration is to fit the hits
to obtain an estimate of the track's parameters.
Between iterations, hits that are unambiguously associated with a track
are removed from consideration for searches for additional hits.  This creates
a smaller collection of hits for the next iteration to consider.
After each iteration, reconstructed tracks are fitted to remove tracks that are likely fakes
and to assess the quality of the remaining tracks.  This filtering for likely fakes uses
the number of hits, the normalized $\chi^2$ of the track, and the compatibility of the 
track originating from a pixel vertex.  Tracks that pass the tightest filtering selection
are labeled ``highPurity''.

The full track reconstruction process uses
a total of six iterations.
The difference between each iteration has largely to do
with its seeds.  Seeds for the first two iterations consist only of 
triplets of pixel tracker hits and pairs of pixel tracker hits with pixel
vertices and the beamspot as an additional constraint.  These first two iterations
find prompt tracks with $\pt > 0.9$~GeV.  The third iteration uses pixel triplets 
as seeds to reconstruct tracks without a \pt~requirement.
The fourth iteration uses both pixel and strip layers as seeds, which allows the 
reconstruction algorithm to reconsrtuct displaced tracks.  
The fifth and sixth iterations use strip pairs as seeds to reconstruct tracks
that do not have pixel hits \cite{track-reco-2010}.

