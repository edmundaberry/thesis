\section{Jets and \met}
\label{sec:jetmet}

As mentioned at the end of Section \ref{sec:forces}, jets are collimated spray of hadrons produced
by the hadronization and fragmentation of quarks and gluons (collectively called ``partons'').
Several algorithms exist \cite{jetography} for the purpose of reconstructing the energy and momentum
of the original parton, using various signatures left by the parton's decay products.
In the case of this analysis, both at the trigger level and at the analysis level, the \akt~algorithm 
\cite{anti-kt} is used with a characteristic radius parameter of $R=0.5$.
At the trigger level, only energy information from the calorimeters is used as an input to the
\akt~algorithm to reconstruct jets.  Jets constructed using calorimeter input alone
are referred to as ``CaloJets''.  CaloJets provide an acceptable jet energy resolution
within relatively little computation time.
At the analysis level, the PF candidates described in Section \ref{sec:reco-pf} are used
as an input to the \akt~algorithm.  Jets reconstructed using the PF algorithm are referred to as ``PFJets''.
PFJets have a better energy resolution than CaloJets \cite{pf-1} (especially at low jet \pt), 
since the PF algorithm measures the momentum of charged hadrons (which carry an average of 90\% of the jet energy)
using the tracker rather than the calorimeters.  However, PFJet reconstruction requires more computation time
than CaloJet reconstruction \cite{cms-jets}.

Some particles, like neutrinos, may be produced in $pp$ collisions without interacting
with the detector.  Neutrinos are neutral, so they do not leave tracks in the inner tracker or muon system,
and they are weakly interacting, so they do not deposit energy in the calorimeters.
These particles may only be detected indirectly, via a vector momentum or energy imbalance
in the transverse plane.  This imbalance (as discussed in Section \ref{sec:coordinates}) is refered to 
as \metvec, and the magnetude of \metvec~is referred to as \met~(MET).
Just as there are many ways of calculating jets, there are many ways of calculating \met.
In the case of this analysis, \met~is calculated using the negative vector sum of the momenta of the 
PF candidates discussed in Section \ref{sec:reco-pf}.  This is expressed in Equation \ref{eqn:met}:
\begin{equation}
  \metvec = - \sum_{i} \vec{p}_{\text{T},i}
  \label{eqn:met}
\end{equation}
where the sum is taken over each PF candidate, $i$, and $\vec{p}_{\text{T},i}$ refers to the 
vector of the transverse momentum of the $i$th PF candidate \cite{cms-met}.

PFMHT, a comparable variable to \met, is calculated using Equation \ref{eqn:met}.  However, in the case of PFMHT, PFJets
are summed over instead of PF candidates.  This variable, therefore, is calculated without input from 
``unclustered energy'' (i.e. without particles from $pp$ collisions that do not form jets).
The amount of unclustered energy in an event is heavily dependent on the number of pile-up
$pp$ interactions, and PFMHT is occasionally used as a less pile-up-dependent substitute for \met.
In this analysis, PFMHT is used exclusively at the trigger level.
