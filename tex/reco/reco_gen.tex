\section{Event generation and simulation}
\label{sec:reco-gen}

Simulated events play an essential role in exotic searches
like the one presented in this thesis.  Simulated events 
are produced using Monte Carlo methods, and the resulting 
datasets themselves are often refered to as Monte Carlo (MC) datasets.
\nomenclature{MC}{Monte Carlo}
These MC datasets allow background Standard Model processes
to be studied in isolation from each other, and they can 
also be combined to provide an approximate estimate of the total 
Standard Model background for the search.  In addition, 
MC datasets allow physicists to study as yet unobserved exotic
signals under various hypotheses, including varying masses, branching
fractions, and other parameters.

In this thesis, all MC datasets are produced by simulating
interacting partons by using a given parton distribution function
(PDF).  Many different parton distribution functions are available,
but this analysis makes use of the {\tt CTEQ6L1} PDFs in all cases
\cite{cteq}.  The hard scattering in these parton-parton interactions is modeled using
event generators.  Different event generators are used to model
different processes in this analysis.  These include
{\sc Pythia}~\cite{pythia},
{\sc Sherpa}~\cite{sherpa},
{\sc MadGraph}~\cite{madgraph-1,madgraph-2}, and 
{\sc POWHEG}~\cite{powheg-st,powheg-w,powheg-1,powheg-2,powheg-3}.  
For all processes in this thesis, the decay and hadronization of the particles
emerging from the hard scatter is modeled by {\sc Pythia}.

In addition, pile-up $pp$ interactions taking place in the same
bunch crossing as the interaction of interest must be modeled in all MC samples.  
Since the number of pile-up interactions in data depends on 
frequently changing conditions at the LHC, the pile-up distribution in all MC samples
must be reweighted to agree with data.  A systematic uncertainty associated with this
reweighting is taken into account in all CMS analyses, and it is discussed in the 
context of this analysis in Section \ref{sec:pile-up-uncertainty}.

In order to compare simulated events to data,
it is also necessary to model the interactions of simulated particles with the CMS detector.
This modeling is performed by the {\sc Geant4} simulation toolkit \cite{geant},
which has been built into the CMS software framework.
The digital signals from the detector's response to these particle 
interactions are also simulated.  These simulated digital signals allow the same 
reconstruction algorithms to be run over simulated events and real events
from the CMS detector \cite{cms-tdr,cms-jinst}.
